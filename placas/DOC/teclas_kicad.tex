\documentclass[11pt,a4paper,twocolomn]{article}
\usepackage{multicol}
\usepackage[utf8x]{inputenc}
\usepackage[spanish]{babel}
\usepackage{ucs}
\usepackage{amsmath}
\usepackage{amsfonts}
\usepackage{amssymb}
\usepackage{makeidx}
\author{Luis Alberto Guanuco}
\title{Combinación Teclas KiCAD}
\begin{document}
\maketitle
\section[twocolomn]{Convinacion de telcas en KiCAD}
La siguiente es una lista de las convinacion de teclas para PCBNew
\begin{multicols}{2}
\begin{description}

\item tecla ?:    Help: this message\\
\item tecla F1:    Zoom In\\
\item tecla F2:    Zoom Out\\
\item tecla F3:    Zoom Redraw\\
\item tecla F4:    Zoom Center\\
\item tecla Home:    Zoom Auto\\
\item tecla Ctrl+U:    Switch Units\\
\item tecla space:    Reset local coord.\\
\item tecla Ctrl+Z:    Undo\\
\item tecla Ctrl+Y:    Redo\\
\item tecla K:    Track Display Mode\\
\item tecla Del:    Delete Track or Footprint\\
\item tecla BkSp:    Delete track segment\\
\item tecla X:    Add new track\\
\item tecla V:    Add Via\\
\item tecla Ctrl+V:    Add MicroVia\\
\item tecla End:    End Track\\
\item tecla M:    Move Footprint\\
\item tecla F:    Flip Footprint\\
\item tecla R:    Rotate Footprint\\
\item tecla G:    Drag Footprint\\
\item tecla T:    Get and Move Footprint\\
\item tecla L:    Lock/Unlock Footprint\\
\item tecla Ctrl+S:    Save board\\
\item tecla Ctrl+L:    Load board\\
\item tecla Ctrl+F:    Find Item\\
\item tecla PgDn:    Switch to Copper layer\\
\item tecla F5:    Switch to Inner layer 1\\
\item tecla F6:    Switch to Inner layer 2\\
\item tecla F7:    Switch to Inner layer 3\\
\item tecla F8:    Switch to Inner layer 4\\
\item tecla F9:    Switch to Inner layer 5\\
\item tecla F10:    Switch to Inner layer 6\\
\item tecla PgUp:    Switch to Component layer\\
\item tecla +:    Switch to Next Layer\\
\item tecla -:    Switch to Previous Layer\\
\item tecla O:    Add Module
\end{description}
\end{milticols}
\end{document}
