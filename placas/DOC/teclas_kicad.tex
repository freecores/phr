\documentclass[11pt,a4paper,twocolomn]{article}
\usepackage{multicol}

\usepackage[utf8x]{inputenc}
\usepackage{ucs}
\usepackage{amsmath}
\usepackage{amsfonts}
\usepackage{amssymb}
\usepackage{makeidx}
\author{Luis Alberto Guanuco}
\title{Combinación Teclas KiCAD}
\begin{document}
\maketitle
\section[twocolomn]{Convinacion de telcas en KiCAD}
La siguiente es una lista de las convinacion de teclas para PCBNew
\begin{multicols}{2}

tecla?:    Help: this message\\
teclaF1:    Zoom In\\
teclaF2:    Zoom Out\\
teclaF3:    Zoom Redraw\\
teclaF4:    Zoom Center\\
teclaHome:    Zoom Auto\\
teclaCtrl+U:    Switch Units\\
teclaspace:    Reset local coord.\\
teclaCtrl+Z:    Undo\\
teclaCtrl+Y:    Redo\\
teclaK:    Track Display Mode\\
teclaDel:    Delete Track or Footprint\\
teclaBkSp:    Delete track segment\\
teclaX:    Add new track\\
teclaV:    Add Via\\
teclaCtrl+V:    Add MicroVia\\
teclaEnd:    End Track\\
teclaM:    Move Footprint\\
teclaF:    Flip Footprint\\
teclaR:    Rotate Footprint\\
teclaG:    Drag Footprint\\
teclaT:    Get and Move Footprint\\
teclaL:    Lock/Unlock Footprint\\
teclaCtrl+S:    Save board\\
teclaCtrl+L:    Load board\\
teclaCtrl+F:    Find Item\\
teclaPgDn:    Switch to Copper layer\\
teclaF5:    Switch to Inner layer 1\\
teclaF6:    Switch to Inner layer 2\\
teclaF7:    Switch to Inner layer 3\\
teclaF8:    Switch to Inner layer 4\\
teclaF9:    Switch to Inner layer 5\\
teclaF10:    Switch to Inner layer 6\\
teclaPgUp:    Switch to Component layer\\
tecla+:    Switch to Next Layer\\
tecla-:    Switch to Previous Layer\\
teclaO:    Add Module\\
\end{multicols}
\end{document}