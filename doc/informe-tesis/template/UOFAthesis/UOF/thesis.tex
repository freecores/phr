
\documentclass[11pt]{report}
%%%%%%%%%%%%%%%%%%%%%%%%%%%%%%%%%%%%%%%%%%%%%%%%%%%%%%%%%%%%%%%%%%%%%%%%%%%%%%%%%%%%%%%%%%%%%%%%%%%%%%%%%%%%%%%%%%%%%%%%%%%%
\usepackage[none,dcucite,abbr]{harvard}
\usepackage{graphicx}
\usepackage{amsmath}
\usepackage{uathesis}
\usepackage{lscape}
\usepackage{longtable}

%TCIDATA{OutputFilter=Latex.dll}
%TCIDATA{LastRevised=Sat Jul 17 17:16:09 1999}
%TCIDATA{<META NAME="GraphicsSave" CONTENT="32">}
%TCIDATA{CSTFile=report.cst}

\newtheorem{theorem}{Theorem}
\newtheorem{acknowledgement}[theorem]{Acknowledgement}
\newtheorem{algorithm}[theorem]{Algorithm}
\newtheorem{axiom}[theorem]{Axiom}
\newtheorem{case}[theorem]{Case}
\newtheorem{claim}[theorem]{Claim}
\newtheorem{conclusion}[theorem]{Conclusion}
\newtheorem{condition}[theorem]{Condition}
\newtheorem{conjecture}[theorem]{Conjecture}
\newtheorem{corollary}[theorem]{Corollary}
\newtheorem{criterion}[theorem]{Criterion}
\newtheorem{definition}[theorem]{Definition}
\newtheorem{example}[theorem]{Example}
\newtheorem{exercise}[theorem]{Exercise}
\newtheorem{lemma}[theorem]{Lemma}
\newtheorem{notation}[theorem]{Notation}
\newtheorem{problem}[theorem]{Problem}
\newtheorem{proposition}[theorem]{Proposition}
\newtheorem{remark}[theorem]{Remark}
\newtheorem{solution}[theorem]{Solution}
\newtheorem{summary}[theorem]{Summary}
\newenvironment{proof}[1][Proof]{\textbf{#1.} }{\ \rule{0.5em}{0.5em}}
\input{tcilatex}

\begin{document}


%TCIMACRO{
%\TeXButton{degree}{\degree{\MSc}%
%}}%
%BeginExpansion
\degree{\MSc}%
%
%EndExpansion
%
%
%
%
%
%
%
%
%
%
%
%
%
%
%
%
%
%
%
%
%
%
%
%
%
%
%
%
%
%
%
%The degree completed.  \PhD or \MSc

%TCIMACRO{
%\TeXButton{dept}{\dept{Department of Chemical and Materials Engineering}%
%}}%
%BeginExpansion
\dept{Department of Chemical and Materials Engineering}%
%
%EndExpansion
%
%
%
%
%
%
%
%
%
%
%
%
%
%
%
%
%
%
%
%
%
%
%
%
%
%
%
%
%
%
%
%
%
%
%
%
%
%
%The Department in which degree completed

%TCIMACRO{
%\TeXButton{field}{\field{Process Control}%
%}}%
%BeginExpansion
\field{Process Control}%
%
%EndExpansion
%
%
%
%
%
%
%
%
%
%
%
%
%
%
%
%
%
%
%
%
%
%
%
%
%
%
%
%
%
%
%
%
%
%
%
%
%
%
%Field of specialization

%TCIMACRO{
%\TeXButton{address}{\permanentaddress{CME 536 \\
%University of Alberta \\
%Edmonton, AB \\
%Canada, T6G 2G6}%
%}}%
%BeginExpansion
\permanentaddress{CME 536 \\
University of Alberta \\
Edmonton, AB \\
Canada, T6G 2G6}%
%
%EndExpansion
%
%
%
%
%
%
%
%
%
%
%
%
%
%
%
%
%
%
%
%
%
%
%
%
%
%
%
%
%
%
%
%Your address

%TCIMACRO{
%\TeXButton{examiners}{\examiners{J. Fraser Forbes, Martin Guay, G. Galileo }%
%}}%
%BeginExpansion
\examiners{J. Fraser Forbes, Martin Guay, G. Galileo }%
%
%EndExpansion
%
%
%
%
%
%
%
%
%
%
%
%
%
%
%
%
%
%
%
%
%
%
%
%
%
%
%
%
%
%
%
%
%Examiners on your committee

%TCIMACRO{
%\TeXButton{convocationseason}{\convocationseason{Fall}%
%}}%
%BeginExpansion
\convocationseason{Fall}%
%
%EndExpansion
%
%
%
%
%
%
%
%
%
%
%
%
%
%
%
%
%
%
%
%
%
%
%
%
%
%
%
%
%
%
%
%
%
%
%
%
%
%
%
%
%When you graduate. (finally...)

%TCIMACRO{
%\TeXButton{quotes}{\frontpiece{
%{\large ```The stars are made of the same atoms as the earth.'
%I usually pick one small topic like this to give a lecture on. 
%Poets say science takes away from the beauty of the stars -- mere gobs of gas atoms. 
%Nothing is ``mere." I too can see the stars on a desert night, and feel them. 
%But do I see less or more? The vastness of the heavens stretches my imagination -- 
%stuck on this carousel my little eye can catch one-million-year-old light. 
%A vast pattern -- of which I am a part --  perhaps my stuff was belched from some forgotten star, 
%as one is belching there. Or see them with the greater eye of Palomar, 
%rushing all apart from some common starting point when they were perhaps all together. 
%What is the pattern, or the meaning, or the *why?* It does not do harm to the mystery to know 
%a little about it. For far more marvelous is the truth than any artists of the past imagined! 
%Why do the poets of the present not speak of it? 
%What men are poets who can speak of Jupiter if he were like a man, 
%but if he is an immense spinning sphere of methane and ammonia must be silent?" \\
%Richard P. Feynman\\
%\vspace{1.0in}
%``The most exciting phrase to hear in science,
%the one that heralds the most discoveries, is not `Eureka!' (I found it!) but `That's funny...'"\\
%Isaac Asimov
%}
%}%
%}}%
%BeginExpansion
\frontpiece{
{\large ```The stars are made of the same atoms as the earth.'
I usually pick one small topic like this to give a lecture on. 
Poets say science takes away from the beauty of the stars -- mere gobs of gas atoms. 
Nothing is ``mere." I too can see the stars on a desert night, and feel them. 
But do I see less or more? The vastness of the heavens stretches my imagination -- 
stuck on this carousel my little eye can catch one-million-year-old light. 
A vast pattern -- of which I am a part --  perhaps my stuff was belched from some forgotten star, 
as one is belching there. Or see them with the greater eye of Palomar, 
rushing all apart from some common starting point when they were perhaps all together. 
What is the pattern, or the meaning, or the *why?* It does not do harm to the mystery to know 
a little about it. For far more marvelous is the truth than any artists of the past imagined! 
Why do the poets of the present not speak of it? 
What men are poets who can speak of Jupiter if he were like a man, 
but if he is an immense spinning sphere of methane and ammonia must be silent?" \\
Richard P. Feynman\\
\vspace{1.0in}
``The most exciting phrase to hear in science,
the one that heralds the most discoveries, is not `Eureka!' (I found it!) but `That's funny...'"\\
Isaac Asimov
}
}%
%
%EndExpansion
%
%
%
%
%
%
%
%
%
%
%
%
%
%
%
%
%
%
%
%
%
%
%
%
%
%
%
%
%
%
%
%
%
%
%
%
%
%
%Frontpage with some quote  by some (in)famous person(s)

%TCIMACRO{
%\TeXButton{dedication}{\dedication{\large To Love, Peace, and the Brotherhood of Man}%
%}}%
%BeginExpansion
\dedication{\large To Love, Peace, and the Brotherhood of Man}%
%
%EndExpansion
%
%
%
%
%
%
%
%
%
%
%
%
%
%
%
%
%
%
%
%
%
%
%
%
%
%
%
%
%
%
%
%
%
%
%
%
%
%
%I dedicate this thesis to ...

%TCIMACRO{
%\TeXButton{author,title}{\title{Constrained Optimization of Nonlinear Chemical Dynamical Systems}
%\author{Sachin Kansal}%
%}}%
%BeginExpansion
\title{Constrained Optimization of Nonlinear Chemical Dynamical Systems}
\author{Sachin Kansal}%
%
%EndExpansion
%
%
%
%
%
%
%
%
%
%
%
%
%
%
%
%
%
%
%
%
%
%
%
%
%
%
%
%
%
%
%
%
%
%
%
%
%
%
%author and title fields. replace with your own, or not...

%TCIMACRO{
%\TeXButton{Front Pages}{\admin  	%
%}}%
%BeginExpansion
\admin  	%
%
%EndExpansion
%
%
%
%
%
%
%
%
%
%
%
%
%
%
%
%
%
%
%
%
%
%
%
%
%
%
%
%
%
%
%
%
%
%
%
%
%
%
%
%generates all the prefatory pages

%TCIMACRO{
%\TeXButton{acknowledgements}{\begin{acknowledgements}
%...
%\end{acknowledgements}
%}}%
%BeginExpansion
\begin{acknowledgements}
...
\end{acknowledgements}
%
%EndExpansion
%
%
%
%
%
%
%
%
%
%
%
%
%
%
%
%
%
%
%
%
%
%
%
%
%
%
%
%
%
%
%
%
%
%
%
%
%
%
%
%Acknowledge everyone, whether they are even aware you exist is inconsequential.

\doublespacing		%abstract has to be double-spaced

%TCIMACRO{
%\TeXButton{abstract}{\begin{abstract}
%......
%\end{abstract}%
%}}%
%BeginExpansion
\begin{abstract}
......
\end{abstract}%
%
%EndExpansion
%
%
%
%
%
%
%
%
%
%
%
%
%
%
%
%
%
%
%
%
%
%
%
%
%
%
%
%
%
%
%
%
%
%
%
%
%
%
%
%
%
%
%
%
%
%the abstract goes into the preceding field between the begin{abstract} and end{abstract}

\onehalfspacing	%everything from here on will be 1.5 spaced

%TCIMACRO{
%\TeXButton{contents and lists}{\tableofcontents
%\listoffigures
%\listoftables
%}}%
%BeginExpansion
\tableofcontents
\listoffigures
\listoftables
%
%EndExpansion
%
%
%
%
%
%
%
%
%
%
%
%
%
%
%
%
%
%
%
%
%
%
%
%
%
%
%
%
%
%
%
%
%
%
%
%
%
%
%
%
%generates lists of tables, figures and the table of contents

\bodyoftext


\chapter{Introduction}

\section{Motivation}

..........

\section{NonLinear Control}

---

\section{Real Time Optimization}

---

\chapter{Problem Definition}

\section{Representative Process}

---

\section{Trajectory Generation}

\subsection{Model Selection}

--- choice of the model which captures the important process behavior yet is
not overly complex and computationally expensive to implement

\subsection{Process Measurements}

--- choice of instrumentation which makes the real-time implementation
possible and yet doesn't require all new instrumentation for the existing
batch process plants

\chapter{Optimization in Flat Output Space}

\section{Introduction}

The dynamic optimization of batch processes ......

%TCIMACRO{
%\TeXButton{B}{\begin{table}[h]  \centering%
%}}%
%BeginExpansion
\begin{table}[h]  \centering%
%
%EndExpansion
\caption{Model Parameters for the CSTR
\label{cstrtable}}\bigskip 

\begin{center}
\begin{tabular}{rlll}
\hline
$\alpha $ & = & $30.828$ & $hr^{-1}$ \\ 
$\beta $ & = & $86.688$ & $hr^{-1}$ \\ 
$\delta $ & = & $3.522\times 10^{-4}$ & $m^{3}.K.KJ^{-1}$ \\ 
$\gamma $ & = & $0.1$ & $K.KJ^{-1}$ \\ 
$T_{in}$ & = & $104.9$ & $^{o}C$ \\ 
$c_{in}$ & = & $5.1\times 10^{3}$ & $mol.m^{-3}$ \\ 
$k_{10}$ & = & $1.287\times 10^{12}$ & $hr^{-1}$ \\ 
$k_{20}$ & = & $9.043\times 10^{6}$ & $m^{3}(mol.hr)^{-1}$ \\ 
$E_{1}$ & = & $9758.3$ &  \\ 
$E_{2}$ & = & $8560.0$ &  \\ 
$\Delta H_{AB}$ & = & $4.2$ & $KJ.mol^{-1}$ \\ 
$\Delta H_{BC}$ & = & $-11.0$ & $KJ.mol^{-1}$ \\ 
$\Delta H_{AD}$ & = & $-41.85$ & $KJ.mol^{-1}$ \\ \hline
\end{tabular}
\end{center}

%TCIMACRO{
%\TeXButton{E}{\end{table}%
%}}%
%BeginExpansion
\end{table}%
%
%EndExpansion

\section{Differentially Flat Systems}

Differential flatness is a concept that applies to underdetermined system of
ODEs. A general underdetermined system ...

\begin{remark}
It is clear from above, that a solution connecting any two generic points in
the original system space can be found. Thus flat systems are controllable.
\end{remark}

\section{Overview of the Process}

The nylon polymerization reaction can most clearly be described in terms of
equivalent or functional groups [Steppan et al., 1987; 1990\nocite{steppan1} 
\nocite{steppan2}] as follows: ...................

The above model is analyzed in the exterior calculus setting of Guay et al.,
1991 and the flat ouputs for the above model, similar to the ones calculated
by Rothfuss et al., 1996, are found to be:

\begin{eqnarray*}
y_{1} &=&T \\
y_{2} &=&\dfrac{c_{in}-c_{A}}{c_{B}}
\end{eqnarray*}

\chapter{Summary and Conclusions}

\section{Contributions of this Thesis}

\section{Recommendations for Industrial Application}

\section{Directions for Future Work}

\subsection{Academic Research Directions}

\subsection{Industrial Research Directions}

\bigskip 

%TCIMACRO{
%\TeXButton{bibliography in contents}{\clearpage\addcontentsline{toc}{chapter}{Bibliography}
%\singlespacing
%}}%
%BeginExpansion
\clearpage\addcontentsline{toc}{chapter}{Bibliography}
\singlespacing
%
%EndExpansion
%
%
%
%
%
%
%
%
%
%
%
%
%
%
%
%
%
%
%
%
%
%
%
%
%
%
%
%
%
%
%
%
%
%
%
%
%
%
%dont forget this if you want the bibliography to show up on the contents page

\bibliographystyle{IFAC}
\bibliography{sachin}
\bigskip 
%Generates the bibliography. You have to specify the source bib files and the biblio style 

%TCIMACRO{
%\TeXButton{Appendices}{\appendix
%}}%
%BeginExpansion
\appendix
%
%EndExpansion
%
%
%
%
%
%
%
%
%
%
%
%
%
%
%
%
%
%
%
%
%
%
%
%
%
%
%
%
%
%
%
%
%
%
%
%
%
%
%The chapters after this tab are all appendices

\end{document}
