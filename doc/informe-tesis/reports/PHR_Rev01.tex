\documentclass[a4paper]{article}
\usepackage[spanish]{babel}
\usepackage[utf8]{inputenc}
\usepackage{acronym}
\usepackage{graphicx}
\usepackage{amsmath}
\author{Luis A. Guanuco\thanks{Universidad Tecnológica Nacional -- Facultad Regional Córdoba}}
\title{Plataforma de Hardware Reconfigurable}
\date{2012}
\begin{document}
\maketitle
\begin{figure}[h]
  \centering
  \includegraphics[width=0.4\textwidth]{images/logov2_ES}
\end{figure}

\begin{abstract} 
  La ciencia de la \emph{Ingeniería} guía al estudiante por el camino de las \emph{ciencias duras}, lo que permite la manipulación de sistemas reales mediante el modelado del mismo. La resolución de los problemas en ingeniería tienen un gran contenido teórico, aunque, en su gran mayoría, tanto el planteamiento como la implementación se llevan acabo en entorno real o medio físico. La presente documentación hará enfoque de la importancia que presenta para los ingenierieros el conocimiento y manejo del \emph{Laboratorio}. También se expone, como un ejemplo de lo redactado, la inserción de la \emph{\ac{PHR}} en ingeniería electrónica, del cuál se podrá obtener conclusiones que sirvan al lector presentandose inquietudez y certezas en el tema.
\end{abstract}

%*****************************************
\section{Introducción}
\label{sec:intro}
El termino \emph{laboratorio} se encuentra presente en varias diciplinas. En ingeniería, se podría clasificar tres grupos fundamentales para los cuales la ídea del Laboratorio toma mayor énfasis, éstos son:
\begin{itemize}
\item[$\cdot$] Desarrollo
\item[$\cdot$] Investigación
\item[$\cdot$] Académico
\end{itemize}

En el \textbf{desarrollo} de un producto, en determinados esenarios, es necesario tomar muestras/datos que sirven al diseño del sistema a implementar. Extendiendose éstos concepto en caso de que se encuentren problemas para la evolución del diseño, donde el laboratorio proporciona valiosa información. El modelado físico resulta beneficioso ya que sobre el mismo se pueden resolver sin los inconvenientes que podría resultar hacerlo sobre el mismo medio, ya sea por la complejidad que sería recrear el ambiente o por simples limitaciones físicas. En la mayoría de las situaciones es necesario la comprobación en el laboratorio como también conciderar sus técnicas fundamentales para la elaboración de un proyecto.
En la \textbf{Invesitagación}, el laboratoría forma parte fundamental en la caracterización de cualquire fenómeno, incluso de aquellos que aún no se tenga información alguna. Como también permite proporcionar información adicional para su posterior documentación.
Ya en el ámbito \textbf{académico}, un estudiante de grado sin los conocimientos de la prácticas o por lo menos una guía difícilmente pueda experimentar satisfactoriamente su actividad de en laboratorio. Aún con los conocimientos de la teoría. El conocimiento de las técnicas a implementar en el laboratorio le permitirán tener un mayor rendimiento como también eliminar errores sistemáticos en la implementación.
Quizá aquí se presentan conceptos muy subjetivos pero son claros ejemplos que sirven de base a fines de obtener concluciones sobre el correcto manejo de una herramienta valiosa como es el \emph{Laboratorio}.
%*****************************************
\section{Observaciones finales}

\appendix{}
\section{Acrónimos}
\begin{acronym}
  \acro{PHR}[PHR]{Plataforma de Hardware Reconfigurable}
  \acro{UTN-FRC}{Universidad Tecnológica Nacional -- Facultad Regional C\'ordoba}
\end{acronym}



\end{document}
